%project-final-report.tex
\documentclass{article}
\usepackage[letterpaper, margin=1.2in]{geometry}
\usepackage{xcolor}
\usepackage{listings}
\usepackage{xparse}

\lstset{language=C,keywordstyle={\bfseries \color{blue}}}

\begin{document}

\title{A Benchmark of Linux System Calls}
\author{Jian Li \and Ding Yuan}
\maketitle

\section{Introduction}

\section{Experiments}

Test 

\subsection{RDTSC vs. clock\_gettime}

A common way of measuring execution time is using C library function \lstinline{clock_gettime(clockid_t clk_id, struct timespec *res)}.
Setting the argument \lstinline{clk_id} to \lstinline{CLOCK_REALTIME} will return the real clock time in nanoseconds precision,
which could be used to measure the execution of system calls.

Intel ia64/x86 architecture provided RDTSC instruction for directly counting CPU cycles. 


\subsection{Environment}

The computer we used to run the benchmark test has an Intel ... CPU with 16GB memory.


\end{document}
